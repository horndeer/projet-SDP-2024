
\section{Introduction}

Dans le cadre de ce projet, nous nous sommes intéressés à la problématique 
de l'assignation optimale des territoires pour les représentants commerciaux 
de Pfizer en Turquie. L'évolution des besoins du marché pharmaceutique et la 
nécessité d'une gestion plus efficace des ressources ont conduit à une 
réévaluation de la répartition des zones géographiques. Le projet vise 
à appliqué les méthodes du cours pour répondre à plusieurs objecctifs : 
la minimisation des distances parcourues, l'équilibre des charges 
de travail entre les représentants, ainsi que la 
limitation des perturbations causées par les changements de territoires.

Ce rapport présente les différentes étapes de notre démarche, depuis 
la modélisation mathématique jusqu'à l'optimisation d'abord mono-objectif 
avec Gurobi, puis multi-objectif par la méthode de contrainte "epsilon". 
Et finalement la méthode UTA pour l'apprentissage des préférences et le
choix de la solution finale parmis l'ensemble de soltions non-dominées données
par la méthode epsilon.